\section{User interface}
\subsection{Overview}
Most programming languages and by extend domain specific languages are parsed
in a structure called abstract syntax trees. The source code first pases
through a lexer who groups strings of characters together into tokens. Since we
implemented a graphical language, the graphical ``source code'' is already
tokenized: we simply group information into predefined graphical primitives who
can be considered the equivalent of a token.
\paragraph{}
When the source code is tokenized, a compiler component called the parser
injects some structure in the token stream by converting the stream into a
tree-structure. The parser knows how to do this since the order of the tokens
contains some cues on how the data is structured.
\paragraph{}
Graphical languages mainly use two paradigms: graph-based and tree-based. In a
graph based language, the user uses two basic concepts: nodes and edges. A
popular example is UML: the class diagram uses classes and the relations
between classes are represented as edges.