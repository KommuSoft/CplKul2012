% !TEX root = dslreport.tex

\subsection{User interface}
\subsubsection{Overview}
The user interface describes the way how the user will communicate with the
system, and thus by extend contains the domain specific language. In our case
we've implemented a graphical language. Furthermore we've implemented a fully
modifiable two-stage compiler.
\subsubsection{Motivation for the graphical language}
Since most users who will use the systems are end-users and thus not
programmers or technical people, some problems might occur with textual
languages. Most end-users nowadays don't use any command line tools anymore and
have no proper understanding of technical textual languages. By implementing a
graphical language, we aim to solve the following problems:
\begin{itemize}
 \itemit{A more semantical granularity} a problem with textual languages is
the granularity, namely the character. Characters don't map to concepts
directly. With graphical languages, we can foresee a primitive for each concept
in the language. Therefore people have a better understanding of their queries.
Furthermore one can easier debug aspects (since there are less issues with
writing an incorrect character sequence).
 \itemit{Giving hints to the user} A graphical language allows us to insert
hints on how queries should be built. By adapting the primitives that only
certain combinations will match, users won't lose time trying these
combinations.
\itemit{Interactive support}Most graphical languages require a tool that somehow
tends to understand the language. These tools can offer interactive support
with the end user in order to build correct queries. This is of course possible
in a textual domain specific language. However these languages don't ``need''
a tool to built them.
\itemit{Limiting Expressiveness}Graphical languages are in general less
expressive than their textual counterparts. Since we implement a domain
specific language, we can see this as an advantage in order to prevent the
users from making mistakes.
\end{itemize}
\subsubsection{Motivation for the Compiler}
\label{sssection:motivation_for_the_compiler}
We've implemented a two-stage compiler where the first stage maintains an
''abstract syntax tree'' consistent with the graphical expression at that
moment. The second stage implements a tiling algorithm looking for specific
patterns and translating them into what we call the intermediate representation.
\paragraph{}
Since we know all the use-cases in advance, we could simply implement
algorithms in the nodes of the abstract syntax tree to deal with translating
the expressions to their algorithmic counterparts. Since the database should
contain all international cities, airports, flights,... we expect the number of
applications to grow. Therefore our implementation can be extended in nearly
all possible ways. Therefore writing a compiler is probably a better idea than
writing a computer program who simply performs some syntactical hacks on the
queries in order to convert them into code for a host language.
\subsubsection{Lexer}
Most programming languages and by extend domain specific languages are parsed
in a structure called abstract syntax trees. The source code first passes
through a lexer who groups strings of characters together into tokens. Since we
implemented a graphical language, the graphical ``source code'' is already
tokenized: we simply group information into predefined graphical primitives who
can be considered the equivalent of a token.
\subsubsection{Parser}
When the source code is tokenized, a compiler component called the parser
injects some structure in the token stream by converting the stream into a
tree-structure. The general tree structure is described by a context-free
grammar (sometimes enriched with a Turing-complete specification language). The
parser knows how to inject structure in the language since he knows the
context-free grammar and uses hints provided by the structure of the source
code.
\subsubsection{Graphical language paradigms}
Graphical languages mainly use two paradigms: graph-based and tree-based. In a
graph based language, the user uses two basic concepts: nodes and edges. A
popular example is UML: the class diagram uses classes and the relations
between classes are represented as edges. The other paradigm is tree based. A
tree based approach is in a sense logical since the second step in a compiler:
the parser also transforms the language into a tree-based equivalent. Since
textual languages are eventually parsed in a tree structure, we can eliminate
some checks that a parser in a textual language should do: for instance
checking if the brackets in the language are matching. Since these constraints
are already forced at language level, it makes concepts easier for both the
programmer in the domain specific language and the developer of the language
itself.
\subsubsection{Beyond the parser}
%TODO: er moet nog een sub bij
\paragraph{Abstract syntax tree}
By working with a graphical language, we can however go beyond the parser. A
first step after the parser is building an abstract syntax tree out of a
concrete syntax tree: in a concrete syntax tree, the tokens are only
represented in the leaves of the tree. When the tree is converted into an
abstract syntax tree, tokens are moved to higher parts in the hierarchy. For
instance in the concrete syntax tree, the tokens \verb+5+, \verb+-+ and
\verb+3+ will all be leaves of the same node. In the abstract syntax tree, the
operator sign \verb+-+ will be the parent of the two integer leaves \verb+5+
and \verb+3+. This step can also be enforced at language level: simply by
making sure that the nodes in the tree contain data (independent from the fact
that they are leaves or not).
%TODO: er moet nog een sub bij
\paragraph{Symbol tables}
After the syntactical analysis, a semantical analysis is performed on a
language. In this step, symbol tables are prepared. A symbol table contains the
name of a variable, method,... together with information about that variable
like it's type. Since in most programming languages variables and methods have
a certain scope, different symbol tables are created for the different scopes.
As a result, nodes in the abstract syntax tree are transformed into pointers
pointing to an entry in the symbol table. Based on the symbol tables,
semantical analysis can perform type checks. For instance the programming
language might force that we can only perform an addition on two integers. We
can modify our graphical language in such a way that it contains hints which
types are accepted and thus boost performance\footnote{If the language hints
or forces certain aspects, users won't lose time by trying these aspects.}.
Furthermore by slightly modifying the tree paradigm, the graphical language can
specify implicit symbol tables. If we allow link-nodes into our language, the
graphical language is based on a Directed Acyclic Graph.The user doesn't have to
specify the same values a second time, but can simply introduce a link
piece pointing to a piece where the data is already specified.
\subsubsection{Proposed graphical language}
We've implemented a graphical language following the tree paradigm. The
graphical language is mainly based on a puzzle. In a puzzle, pieces must fit
into the gaps other pieces provide. Since each puzzle piece fills one gap and
provides zero or more gaps, a tree structure emerges.
%TODO: er moet nog een sub bij
\paragraph{Polymorphism and overloading}
How can the graphical language hint (or even force) the use of a certain piece
to fill a certain gap? In the physical world the gaps follow a certain pattern.
Only pieces who are compatible with this pattern, can fit into the gap.
\subparagraph{}
The problem however with this approach is that the number of potential pieces
is in most cases limited to one. Sometimes however, it is possible that there
are different solutions to fill a gap. Most programming languages solve this
problem with polymorphism. We will need a mechanism to allow polymorphism.
Another aspect some puzzles use is graphical coherence: the resulting puzzles
must form a picture, not a colorful ensemble of dots.
\subparagraph{}
We adapted this concept to introduce polymorphism in our graphical language.
Each gap is surrounded by a sequence of colors. Each piece is filled with a
sequence of colors. If a puzzle piece has at least one color in common with the
colors surrounding the gap, the user of the language is allowed to use this
piece\footnote{It is of course possible that there are still constraints who
can't be expressed by a type system. The ``compiler'' of the domain specific
language will have to check on these additional constraints.}.
\subparagraph{}
Since the pieces only have to share one color, one can look to this relation as
a many-to-many relation: each gap allows several types of pieces. On the
other hand a piece can fit into gaps with different type constraints. Therefore
this aspect is stronger than polymorphism: also overloading is supported since
the types of the ``parameters'' of a puzzle piece are somehow dynamic.
\subparagraph{}
A constraint on the use of colors is that the number a person can distinct is
limited. Research points out that humans with normal color perception
(trichromatic) can distinct around 36 colors. Since domain specific languages
however in general and more specific this language is limited in the amount of
polymorphism and overloading we assume this system will be sufficient.
Furthermore the programmer of the Domain Specific Language can of course
introduce other constraints. In that case, the user can't rely on the visual
hints.
\paragraph{Implementation}
\subparagraph{Domain Specific Language}
The implementation of the graphical domain specific language is quite straight
forward. Since we model the language as a tree, we use this paradigm to
represent the language internally. We can of course easily put constraints on
the structure of the tree by implementing a method in each type of node
(flight, booking,...) who checks if his/her children are valid. A link piece is
nothing more than a piece who maintains a reference to another piece.
Furthermore it adapts some of it's properties (like his children), to the
properties of his reference (thus the children of the reference). Most nodes
hold three types of information: their name, their subpieces, and a key-value
table.
\subparagraph{The Intermediate Representation}
Turning the domain specific language into an intermediate representation (our
equivalent to machine code), is done like real compilers: by a tiling
algorithm. A tiling algorithm performs pattern matching on the abstract syntax
tree and replaces this tree by a sequence of target-patterns (comparable to
for instance regular expression replacement). Since the abstract syntax tree
contains information (the name of an airline,...) we have to bind that
information into a data structure. Therefore we use a dictionary data
structure. Variables in the structure are represented by a string. A value is a
simple object and can theoretically be anything. The tiling algorithm looks for
all patterns defined in the implementations. Each pattern contains a tree
holding information about the structure of the pattern (the tree of puzzle
pieces) and data about which data should get linked to the dictionary. Besides
information about binding data, one can also specify additional constraints
like a specific index of a node relative to his parent. Some nodes can be
optional (if they contain information, they slightly modify the execution of
the query). This can be specified by the constraints of the language.
Furthermore each pattern contains a construction method. Each time the binding
of a certain pattern succeeds, the method is called. This method thus turns the
tree together with bounded data into the intermediate representation.
We conclude with a small example of one pattern:
\begin{lstlisting}
using System.Collections.Generic;
using DSLImplementation.UserInterface;
using DSLImplementation.IntermediateCode;

namespace DSLImplementation.Tiling {

  [TilePatternAttribute]
  public class RequestAddCityTilePattern : TilePatternBase {

    private static readonly Tree<TypeBind> bindtree = new
      new Tree<TypeBind>(typeof(AddPiece),new Tree<TypeBind>(
      new TypeBind(typeof(CityPiece),0,"name","cityname"),
      new TypeBind(typeof(CountryPiece),"name","countryname")));

    public RequestAddCityTilePattern () : base(bindtree) {
    }

    protected override IRequest InternalToTransferCode (IPuzzlePiece root,
      Dictionary<string, object> bindings) {
      return new RequestAddCity(new City(
        (string) bindings["cityname"],
        new Country((string) bindings["countryname"])));
    }

  }

}
\end{lstlisting}
We use the attribute \texttt{TilePatternAttribute} to mark a tile pattern as an
active one. This means that the program will use reflection to look for
patterns (otherwise we should develop a bootstrap procedure to add new
patterns to the collection, which will become complex if add a new executable
to the project). Furthermore we define a tree-structure called
\texttt{bindtree}. The bindtree is a tree of constraints. For instance in order
to add a city, we will need an add piece. The first child of the add piece
should be a city piece and one of the children of that city piece should be a
country piece. Furthermore we will extract the name of the citypiece and add
this to the dictionary as \texttt{``cityname''}, an analog action happens with
the name of the country. When the pattern matches and the information has been
added to the dictionary. We will simply compile an intermediate representation
described in the method called \texttt{InternalToTransferCode}.