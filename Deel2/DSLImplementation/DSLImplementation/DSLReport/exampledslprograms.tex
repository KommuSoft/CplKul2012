% !TEX root = dslreport.tex

\section{Example DSL programs}
In this section, we show some DSL example programs. %We have also put the example program files in the folder TODO:folder, because some example programs may be a bit unreadble in this document.

\begin{figure}[H]
	\centering
	\includegraphics[width = \textwidth]{../../../Voorbeeldprogrammas/programlayout.png}
	\caption{The graphical editor}
	\label{ex:DSLlayout}
\end{figure}
\noindent In the editor shown in figure \ref{ex:DSLlayout}, we can create our programs. This can be done by selecting the appropriate puzzle piece and then selecting the right block in the striped area. In that way, a chain of blocks is obtained. An example of such a chain of blocks is given in the next figure. In fact, it are three chains of blocks, but it now permits to give another illustration of the linking feature.
\begin{figure}[H]
	\centering
	\includegraphics[width = \textwidth]{images/addflight2.pdf}
	\caption{The lay-out of a program}
	\label{ex:generalexample}
\end{figure}
Now, we will work out a small example in detail.

\subsection{Adding a country to the database}
As seen in figure \ref{ex:DSLlayout}, the color of the \texttt{run}block matches two puzzle pieces, namely \texttt{query} and \texttt{add}.
In this example, we want to add something, so we select the \texttt{add} puzzle piece and click on the \texttt{run}block Then, we can select the country puzzle piece and click on the \texttt{add}block. Adding another element instead of this, is just selecting another puzzle piece. The result of adding a country is seen below.
\begin{figure}[H]
	\centering
	\includegraphics[width = 0.6\textwidth]{../../../Voorbeeldprogrammas/AddCountryInitial.pdf}
	\caption{Adding a country}
	\label{ex:addCountryInitial}
\end{figure}
\noindent We can also give the country a name by using the \texttt{Edit information} button on top. When this button is selected and you click on a block in the chain, the properties can be changed. The final result is given in the next figure.
\begin{figure}[H]
	\centering
	\includegraphics[width = 0.6\textwidth]{../../../Voorbeeldprogrammas/AddCountry.pdf}
	\caption{The final program for adding a country with name 'Germany'}
	\label{ex:addCountry}
\end{figure}
As a remark: when editing information, it is important to push enter after filling in a value. Otherwise the values won't be saved.

\subsection{Adding a flight to the database}
As said before, we can add several sorts of elements to the database. A bigger example is adding a flight to the database. The program is shown below.
\begin{figure}[H]
	\centering
	\includegraphics[width = \textwidth]{images/addflight2.pdf}
	\caption{The final program for adding a flight}
	\label{ex:addFlight}
\end{figure}

\subsection{Fetching the cities in a country}
It is possible to do more than only adding elements to the database. Namely, it is possible to get things from the database. For example, finding all the cities in a given country 'Belgium' can be done by the following DSL program.
\begin{figure}[H]
	\centering
	\includegraphics[width = 0.6\textwidth]{../../../Voorbeeldprogrammas/QueryCity.pdf}
	\caption{The final program for finding the cities in belgium}
	\label{ex:getCities}
\end{figure}