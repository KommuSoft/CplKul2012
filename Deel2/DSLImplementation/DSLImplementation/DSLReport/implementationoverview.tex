% !TEX root = dslreport.tex

\section{Implementation overview}
Our DSL isn't really based upon an API. We've implemented our own compiler, this is explained in Subsubsection \ref{sssection:motivation_for_the_compiler}. Of course in order to execute the programs written in the DSL we need some communication with the database. This is done with the library \texttt{Npgsql} and more in detail the classes \texttt{IDbConnection} and \texttt{IDbCommand}. However our usage of this API is a very primitive, we only need to open, close and execute queries on the database. When we need transform data from the database to the objects from the database frontend we need a system that provides this operations in its API. This is done with the class \texttt{IDataReader} which provids operations such as \texttt{GetString}, \texttt{GetInt32}, ... these methods transform the data of the cell into the desired type. Another interesting method is \texttt{GetOrdinal}, this method returns the position of the column with the given name. This way we can extract information from the database.