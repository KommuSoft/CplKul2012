% !TEX root = dslreport.tex

\subsection{Adapter}
The adapter layer operates as an intermediate between the user interface and the database. This way, the user interface doesn't need to communicate directly with the database frontend. It can make complete abstraction of it. The adapter classes pass the requests from the user interface to the database and return results of queries on the database to the user interface.

The adapter layer has three types of classes. A first type represents the data of the database, the other two types represent the operations that can be done on the database, i.e. adding data and fetching data. For example, the class \texttt{Airport} has two constructors. One with only the code of the \texttt{Airport} and another one with the possbile data that can be stored in an \texttt{Airport}. The first one is used to specifiy an already existing \texttt{Airport}, the other one is used to put data in the database. The latter one should be used for the class \texttt{RequestAddAirport}. This class has a method \texttt{execute} which puts the property \texttt{Airport} into the database. The class \texttt{RequestGetFlights} uses only the code of the \texttt{Airport}, so the first constructor can be used.

The adapter layer does also some consistency checking. The adapter layer has to transform information from the user interface into information that is usable for the database frontend. Therefore it also has to do some database requests. When these requests returns an unexpected answer an inconsistency is found and this is reported to the user interface.