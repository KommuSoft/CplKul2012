\section{Design description}
\subsection{The semantic model}
\subsection{DSL constructs}
If we look at the graphical editor on figure \ref{ex:DSLlayoutinDD}, we can see almost all of the DSL constructs.
\begin{figure}[H]
	\centering
	\includegraphics[width = \textwidth]{../../../Voorbeeldprogrammas/programlayout.png}
	\caption{The graphical editor}
	\label{ex:DSLlayoutinDD}
\end{figure}
\noindent On top of the figure are the puzzlepieces mentioned. There are two special ones, namely \texttt{Query} and \texttt{Add}. This is because building all programs have to start with selecting one of them.  Now, we give a short description.
\begin{description}
 \item[Query] With the \texttt{Query} puzzlepiece, we can request information from the database.
 \item[Add] With the \texttt{Add} puzzlepiece, we can put elements in the database.  
\end{description}
The puzzlepieces that are next placed in \texttt{Query} or \texttt{Add} will decide what must be added or queried.
All the other puzzlepieces represent concepts in the domain. They also have attributes than can be initialized by using the \texttt{Edit information} button. It is also possible that those puzzlepieces contain other puzzlepieces. 
Another important DSL construct are links. They can be used to copy the values of attributes from one element to another. An example is given in the figure below.
\begin{figure}[H]
	\centering
	\includegraphics[width = \textwidth]{../../../Voorbeeldprogrammas/LinkPieceExample.pdf}
	\caption{An illustration of making use of links}
	\label{ex:linkpieceexample}
\end{figure}
First we have added a country Japan. After that, we wanted to add a city that is located in Japan. What we did, was putting a link from the Japan block at the bottom of figure to the place where the city Tokio request a country. In that way, you are sure to make no typing errors or other inconsistencies.

