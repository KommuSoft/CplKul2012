\section{Design description}
\subsection{The semantic model}
\subsection{DSL constructs}
If we look at the graphical editor on figure \ref{ex:DSLlayoutinDD}, we can see almost all of the DSL constructs.
\begin{figure}[H]
	\centering
	\includegraphics[width = \textwidth]{../../../Voorbeeldprogrammas/programlayout.png}
	\caption{The graphical editor}
	\label{ex:DSLlayoutinDD}
\end{figure}
\noindent On top of the figure are the puzzlepieces mentioned. There are two special ones namely \texttt{Query} and \texttt{Add}. This is because all programs have to start with one of them.  Now, we give a short description.
\begin{description}
 \item[Query] With the \texttt{Query} puzzlepiece, we can request information from the database.
 \item[Add] With the \texttt{add} puzzlepiece, we can put elements in the database.  
\end{description}
All the other puzzlepieces represent concepts in the domain. They also have attributes than can be initialized.
Another DSL construct are links. They can be used to copy the values of attributes from one element to another.
